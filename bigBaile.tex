\documentclass[a4paper]{article}

\usepackage[portuguese]{babel}
\usepackage[utf8]{inputenc}
\usepackage {algpseudocode}
\usepackage{amsmath}
\usepackage[portuguese,onelanguage,lined]{algorithm2e}
\usepackage{graphicx}
\usepackage{setspace}
\usepackage{natbib}
\usepackage[a4paper,rmargin=2cm,lmargin=3cm,tmargin=3cm,bmargin=2cm]{geometry}
\usepackage{subfig}

\title{Organizando um Baile: Solução do Problema da Organização de uma festa}
\author{Erick Grilo, Max Fratane   Matheus Prado e Vitor Araújo}

\SetKwInput{Entrada}{Entrada}
\SetKwInput{Resultado}{Resultado}
\SetKwFor{Enquanto}{Enquanto}{faca}{fim-enquanto}
\SetKwIF{Se}{Senao Se}{Senao}{Se}{Entao}{Senao}{fim-se}



\begin{document}
\begin{flushright}
\thispagestyle{empty}
\includegraphics[width=.2\textwidth]{IC-UFF.pdf}
\end{flushright}

\begin{center}
\vfill
\vspace{-7em}
\emph{\Large Organizando um Baile: Solução do Problema da Organização de uma festa}
\begin{flushright}
\vspace{1em}
\makebox[.5\textwidth][l]{\parbox{.5\textwidth}{
\vspace{2em}
Erick Grilo\\ 
Max Fratane\\ 
Matheus Prado\\
Vitor Santos\\
}}
\end{flushright}
\vfill
\end{center}

\newpage

\section{Introdução}
\paragraph{} O problema consiste em que o professor Stewart foi contradado para prestar um serviço de consultoria para o presidente de uma determinada empresa que deseja realizar uma festa e nós vamos ajudá-lo. A empresa possui uma hierarquia tal que a relação de hierarquia forma uma árvore 
cuja raiz é o presidente da empresa. O departamento do RH classificou cada um dos empregados com uma classificação de convivialidade (que é um número real). Para a festa ser proveitosa o máximo para todos que forem, o presidente não quer que ambos um funcionário e seu supervisor vão simultaneamente.
\paragraph{} Ao professor Stewart, é dada uma árvore
onde cada nó da árvore é um funcionário da companhia, que possui um nome e um valor de convivialidade, onde o pai desse nó é o seu supervisor imediato. O objetivo é criar um algoritmo que cria uma lista de convidados que maximiza a soma dos valores de convivialidade dos convidados.

\section{O Algoritmo}\label{sec:rec}
\paragraph{} A ideia do algoritmo consiste em solucionar o problema para a árvore de funcionários, com dois casos: caso a árvore tenha somente um nó, e caso a árvore tenha mais de um nó. Dessa forna, a partir do enunciado, temos:


\begin{equation}
	Baile(tree) = \begin{cases}
		\text{convivialidade(\textit{tree}),}  \text{caso tree não tenha filhos}\\
		\text{MAX}  \begin{cases}
			\sum \text{Baile(\textit{t}),}  \forall t \in \text{filho(\textit{tree})}\\
			\text{convivialidade(\textit{raiz})} + \sum  Baile(t), \forall t \in \text{neto(\textit{tree})}
			\end{cases}
		\end{cases}
\end{equation}
\paragraph{}Onde o caso base é o caso de \textit{tree} ser uma árvore com um só nó e a relação de recorrência consiste no caso de tree ter filhos: temos que avaliar então, dentre todos os filhos da árvore de entrada, quais serão os nós da árvore que maximizarão o somatório de convivialidade, tomando a precaução de não permitir que um empregado e seu supervisor imediato possam ir ao baile. Nesse caso, na primeira linha da parte do MAX, é avaliado o caso dos filhos de tree (caso o nó da árvore atual vá à festa), e na segunda linha, a convivialidade do nó imediatamente acima de tree e os filhos dos seus filhos, ou seja, o funcionário do funcionário da empresa. Dessa forma, é evitado que um funcionário e seu supervisor imediato possam ir ao baile. A função funcionário retorna o funcionário de um determinado nó, enquanto a função conviviabiidade retorna o valor de convivialidade de um determinado nó.\\
\paragraph{}Da relação de recorrência apresentada acima, temos o seguinte pseudo-código, que ilustra como o algoritmo funciona:

\begin{algorithm}
\DontPrintSemicolon
\Entrada{tree: uma árvore com os funcionários da empresa}
\Saida{A lista de convidados tais que a soma dos valores de convivialidade de seus elementos seja a máxima possível}
Resolver(tree):\;
\Begin{
Baile(tree,lista)\;
\Return lista}
\caption{{\sc Resolver}}
\label{alg:resolver}
\end{algorithm}

\paragraph{}Onde, a função \emph{resolver} retorna a lista desejada, enquanto a função principal responsável por procurar a sequência de valores que maximizam a soma das convivialidades (\emph{baile}) é chamada no seu escopo. Tal função tem seu comportamento descrito a seguir:\\

\begin{algorithm}
\DontPrintSemicolon 
\Entrada{tree: uma árvore com os funcionários da empresa, Lista: lista de convidados, inicialmente vazia}
\Saida{O maior valor de convivialidade, de acordo com o caso}

Baile(tree,lista):\;
\Begin{
    \uIf{$\text{tree já foi visitado}$}{
	$lista \gets \text{append(lista,lista(tree))}\text{//junta a lista de entrada com a lista armazenada do nó}$\;
	\Return o valor armazenado em tree
     }
     $visitada(tree) \gets True$//o nó tree é marcado como visitado\;
    \uIf{$\text{tree não possui filhos}$}{
      $lista \gets \text{adicionar funcionario(tree)}$//funcionário(tree): retorna o funcionário desse nó da árvore.\;
      $listaArmazenada \gets insert(listaArmazenada, lista(tree))$//armazena lista de tree\;
      $crArmazenado \gets insert(crArmazenado, convivialidade(tree))$\;
      \Return convivialidade(tree)//convivialidade(tree): retorna o valor de convivialidade desse nó da árvore.\;
    }
    \Else{
      $A \gets 0.0$\;
      $B \gets \text{convivialidade(tree)}$\;
      \ForEach{filho $x \in \text{filho(tree)}$}{
      $A \gets A + Baile(x,listaA) $//listaA: lista auxiliar\;
           \ForEach{filho $y \in \text{filho(x)} $}{
           $B \gets B + Baile(y,listaB) $//listaB: lista auxiliar\;
            }
       }
      \If{$B > A$}{
	$listaB \gets insert(listaB, funcion\acute{a}rio(tree))$\;
	$listaArmazenada \gets insert(listaArmazenada, listaB)$\;
	$crArmazenado \gets insert(crArmazenado, B)$\;
	$lista \gets append(lista, listaB)$\;
	\Return B\;
	}
	$listaArmazenada \gets insert(listaArmazenada, listaA)$\;
	$lista \gets append(lista, listaA)$\;
	$crArmazenado \gets insert(crArmazenado, A)$\;
	\Return A\;
	}
} 
\caption{{\sc Baile}}
\label{alg:baile}
\end{algorithm}
\newpage
\section{Exemplo}
\paragraph{}Suponha que a empresa em questão seja uma pequena empresa de informática, e como estamos em junho, A empresa pediu para usarmos o algoritmo para resolvermos o problema em questão referente à uma festa junina. Dessa forma, se um funcionário for, nem seu subordinado imediato nem o seu superior imediato poderão ir. A árvore abaixo ilustra a organização dos funcionários dessa empresa:\\

\begin{figure}[!htb]
\centering
\includegraphics[scale=0.69]{arvore2.png}
\caption{Exemplo de uma árvore que representa funcionários de uma empresa.}
\label{arvore}
\end{figure}

\paragraph{}Ao se executar o algoritmo, temos que o mesmo verificará que a árvore de entrada não é uma árvore sem filhos. Logo, ele segue o segundo caso descrito na relação de recorrência, que verifica o máximo dos valores entre todos os filhos de \emph{tree} ou o valor de convivialidade da raiz com o dos netos (assim, respeitando a restrição).\\

\paragraph{}Ao entrar no caso de \emph{tree} (a raiz) ter filhos, para cada filho de \emph{tree}, a função \emph{baile} é chamada recursivamente a fim de armazenar o valor de convivialidade de cada um dos filhos de \emph{tree} em uma variável A qualquer. Em seguida, o mesmo ocorre, só que agora para os netos de \emph{tree}, armazenando a soma dos valores em B (que já recebera o valor de convivialidade de \emph{tree}). Ou seja, Quando \emph{Baile}(69) é chamada, temos que avaliar o máximo entre (\emph{Baile}(15) + (\emph{Baile}(21) + \emph{Baile}(18) e (69 + \emph{Baile}(5) + \emph{Baile}(11) \emph{baile}(14)), que são respectivamente o somatório dos valores do filho da raiz com o seu valor mais o valor de seus netos.\\

\paragraph{} O algoritmo então verifica primeiro para cada um dos filhos da raiz: chama recursivamente \emph{Baile}(15), que por sua vez chama recursivamente \emph{baile} para seus filhos, \emph{baile}(5) e \emph{baile}(1). Como \emph{baile}(1) e \emph{baile}(5) são folhas, o seu valor de convivialidade é retornado. Dessa chamada recursiva, é retornado o máximo entre 15 (o valor da raiz da chamada) e a soma de seus filhos (5+1). O valor retornado é 15\\

\paragraph{}Em seguida, o algoritmo chama \emph{baile} para o outro filho da raiz, o nó 25, que por sua vez chama recursivamente \emph{baile} para seu único filho, 14, que por sua vez, chama \emph{baile} para cada um de seus filhos (6,6 e 6). Como esses três nós são folhas, os seus valores de convivialidade são retornados. Dessa chamada recursiva, é retornado o máximo entre 14 e 18 (6+6+6). Logo, o valor 18 é retornado. Dessa forma, para a chamada do nó 25, retorna o valor da raiz da chamada (25) somado com o somatório dos seus netos (18), que é justamente o segundo caso da relação de recorrência. porém, o valor da raiz da árvore é maior que 25. Logo, o valor final dessa chamada acaba por ser 69+ 18.\\

\paragraph{}Por último, o último filho da raiz é visitado: \emph{baile}(18) é chamado recursivamente, que, como não possui filhos, acaba por retornar o valor de convivialidade desse nó (o próprio 18).Logo, temos:\\
n1 = 15+25+18 e n2 = 5 + 1 + 69 + 6+6+6.
\emph{baile}(69) = MAX(58,93). = 93. Logo, os nós que vão compor a lista de convidados são os funcionários que são representados por esses nós, cuja soma maximizou o valor de convivialidade respeitando a restrição de que, se um convidado vai, seu supervisor imediato não vai (e vice-versa).\\

\newpage
\section{Prova da Corretude do Algoritmo}

\paragraph{}A função que resolve o problema é dada pelo pseudocódigo definido em \ref{alg:resolver} (\emph{resolver}). Porém, a função que efetivamente resolve e processa os dados para solucionar o problema é a ~\ref{alg:baile} (\emph{Baile)}. Dessa forma, ao provarmos a corretude da função \emph{Baile}, provamos a corretude de toda o algortimo que soluciona o problema.\\

Seja \emph{Baile(T)}: A chamada da função \emph{Baile} tal como descrita na relação de recorrência na seção \ref{sec:rec} ou seja, \emph{Baile}(T) retorna o valor máximo do somatório das convivialidades dos seus nós. A prova será feita por indução forte estrutural na árvore de entrada para o problema.\\

\textbf{(i) \underline{Base da indução:}} \emph{Baile}(T$_{0}$) é verdadeira?\\
\qquad Seja \emph{Baile}(T$_0$) : O valor máximo da soma de convivialidade da árvore que possui apenas um nó (sem filhos). Como só existe um único nó a ser avaliado, temos que o seu valor de convivialidade (por definição) é o que maximiza o somatório dos valores de convivialidade da árvore. Portanto, a lista retornada por \emph{Resolver} só possui o funcionário representado por esse nó como um convidado. Logo, temos que \emph{Baile}(T$_{0}$) vale.\\

\textbf{(ii) \underline{Hipótese de Indução:}} Sejam  T$_{1}$,T$_{2}$,T$_{3}$,\ldots,T$_{n}$ sub-árvores formadas a partir da raiz (árvore de entrada para o problema) cujas raízes estão no k-ésimo nível, para k = 1, 2, \ldots, n. Logo, supor que \emph{Baile}(T$_k$) vale, para k = 1, 2, \ldots, n. Por causa dessa suposição, temos que a relação de recorrência também passa a valer para sub-árvores de T$_k$, ou seja, árvores T$_{k_q}$ que são sub-árvores de uma T$_k$, com q = 1,2,\ldots,m.\\

\begin{figure}[!htb]
\centering
\includegraphics[scale= 0.48]{arvorecorretude.png}
\caption{Exemplo de uma árvore com até n níveis e m filhos, onde o limite de m pode variar de nível em nível, uma vez que a árvore é n-ária}
\label{arvorecorretude}
\end{figure}

\textbf{(iii) \underline{Passo Indutivo:}} Provar válido para qualquer árvore do nível \emph{Baile}(T$_{n+1}$), uma T$_{n+1_q}$ qualquer, q = 1,2,\ldots,m.\\
Ao entrarmos nesse passo, fixemos para uma T$_{n+1_q}$ qualquer tal como na árvore acima, sem perda de generalidade. A partir daí, temos:
pela hipótese de indução, \emph{Baile}(T$_k$) é o valor máximo da convivialidade para uma sub-árvore do nível k, T$_{k}$, para k = k+1,k+2,\ldots,n. Dessa forma, temos que \emph{Baile}(T$_k$) é o valor máximo do somatório do valor de convivialidade de T$_k$. Ao considerarmos que \emph{Baile}(T$_n$), vale (pela hipótese de indução), se \emph{Baile}(T$_n$) possui filhos, chegamos em \emph{Baile}(T$_{n+1}$) recursivamente. Logo, temos que analisar dois casos distintos:

\textbf{I)} Uma árvore qualquer do nível T$_{n+1}$, a T$_{n+1_q}$ que estamos analisando não possui filhos : pela base da indução, nota-se que o valor de convivalidade máximo dessa árvore, é o seu próprio valor de convivialidade.

\textbf{II)}Uma árvore qualquer do nível T$_{n+1}$ possui filhos:
\paragraph{}Seja n$_1$ = $\sum \text{\emph{Baile(t)}}, \forall t \in \text{filho(T$_{n+1_q}$)}$. Pela hipótese de indução, temos que \emph{Baile}(T$_k$) vale, para k = 1,2,\ldots, n. Como extendemos isso para qualquer filho de T$_k$ (logo, para qualquer filho de T$_n$), vale também para o caso de T$_{n+1_q}$ (que é alcançado recursivamente na chamada \emph{Baile}(T$_n$), onde T$_{n+1}$ é filha de T$_n$.  Portanto, n1 é o somatório dos valores dos filhos de uma árvore qualquer do nível T$_{n+1}$, a T$_{n+1_q}$ genérica introduzida no passo indutivo.

\paragraph{}Agora, seja n$_2$ = convivialidade(raiz) + $\sum \text{\emph{Baile(t)}}, \forall t \in \text{neto(T$_{n+1}$)}$. Para o caso de T$_{n+1}$, faz-se a raiz sendo a própria T$_{n+1}$ que é raiz da T$_{n+1_q}$. Como no parágrafo acima, vimos que vale para um filho de uma árvore no nível de T$_{n+1}$ (por valer para qualquer neto de qualquer árvore de T$_{n}$, pela hipótese), vale para a árvore que fixamos, T$_{n+1_q}$. Logo, se a raiz nesse caso é a árvore pai da árvore do nível T$_{n+1}$ fixada, a T$_{n+1_q}$, n2 é o valor do somatório da convivialidade de T$_{n+1_q}$ com os seus netos.\\
 
\paragraph{}Logo, por n$_1$ e n$_2$, o valor de convivialidade máximo da árvore original que foi dada como entrada para o problema (T), pois como \emph{Baile}(T) vale para qualquer sub-árvore até o n-ésimo nível (hipótese de indução), pode ser encontrada pela seguinte fórmula:\\
\emph{Baile(T)} = \emph{Baile(T$_n$)} + \emph{Baile(T$_{n+1}$)}, onde \emph{Baile(T$_n$)} é o baile para todas as sub-árvores de T (a árvore original), que pela hipótese de indução,vale.\\
\paragraph{}Porém, podemos reescrever \emph{Baile(T$_{n+1}$)} como sendo:
\begin{center} 
\textbf{\emph{Baile(T$_{n+1}$)} = convivialidadeMaxima(T$_{n+1_q}$) =  MAX(n1,n2),} \\$\forall T_{n+1_q}$ filhas de qualquer árvore do nível T$_{n+1}$.
\end{center}
\paragraph{}Temos que a função acima culmina justamente na relação de recorrência que foi encontrada. Ao provarmos, sem perda de generalidade, que \emph{Baile} vale para uma \emph{Baile(T$_{n+1_q}$} qualquer, generalizamos acima novamente. Como MAX retorna o valor máximo entre dois valores, temos que convivialidadeMaxima retorna o valor máximo dentre os possíveis valores máximos que T$_{n+1_q}$ pode ter, não retornando um resultado inválido para a solução do problema da árvore de entrada T. Logo, para qualquer sub-árvore de T$_{n+1_q}$,\emph{Baile(T$_{n+1}$)} vale.
\newpage

\section{Prova da Complexidade do algoritmo}
\paragraph{}Após uma série de testes, a primeira versão do algoritmo (sem o uso da noção de programação dinâmica, nesse caso, verificando se o algoritmo já avaliou o valor de convivialidade do nó), havíamos chegado em uma suposição de que o algoritmo parecia estar com a complexidade em $O(n^2)$. Após uma segunda versão do algoritmo ter sido gerada (com a noção de marcar o nó que já foi visitado), chegamos, através de testes e medição de quantidade de nós visitados à complexidade de $O(n)$.\\
\paragraph{}Pelo método da suposição e verificação, temos que a complexidade de \emph{Baile} está em $O(n)$. A prova será feita por indução em n. Queremos provar que a complexidade de Baile é $O(n)$.\\

\textbf{(i) \underline{Base da indução:}} \emph{Baile} com a árvore de de tamanho 1 é $O(1)$ ?\\
\qquad Seja P(n): Baile($T_n$) possui complexidade n. Como só existe um único nó a ser avaliado, temos que o número de iterações que o algoritmo faz é apenas 1. Portanto, a compexidade do algoritmo nesse caso é $O(1)$, com n = 1.\\

\textbf{(ii) \underline{Hipótese de Indução:}} Supor válido P(k), para $1\leq k \leq n$ (indução forte).\\

\textbf{(iii) \underline{Passo Indutivo:}} Provar válido P(n+1); ou seja, para uma árvore de entrada que possui n + 1 nós, a complexidade de \emph{Baile}(T$_{n+1}$) é $O(n+1)$ = $O(n)$.\\
A partir do algoritmo, é possível deduzir:\\
1 chamada para \emph{Baile(filho(n+1))};\\
p chamadas de \emph{Baile(neto(n+1))}. Logo, como \emph{Baile(filho(n+1))} visita recursivamente \emph{Baile(neto(n+1))},temos:\\
\emph{Baile(filho(n+1))} $\in O(p)$ e \emph{Baile(neto(n+1))} $\in O(n)$. Logo, como ($O(p) + O(n)$) = $O(n)$ (pois n $>$ p), \emph{Baile(n+1)} $\in O(n)$.






\end{document}